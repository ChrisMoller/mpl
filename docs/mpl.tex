\input table
\lineskip=6pt
\parskip=12pt
\parindent=0pt

\font\bigsl = cmsl10 scaled \magstep 2
\font\vbigsl = cmsl10 scaled \magstep 3
\font\vbigbx = cmbx10 scaled \magstep 3
\font\bigbx = cmbx10 scaled \magstep 2
\font\bigrm = cmr10 scaled \magstep 2

\newcount\fn
\def\fnote{\global\advance\fn1
    \footnote{$^{\the\fn}$}}

\newcount\captnr
\def\caption{\global\advance\captnr1
    {Figure \the\captnr}. }

{\vbigsl MPL}  {\vbigbx --- Mathematical Programming Language}\fnote{Or, My
Programming Language---take your pick.}

\bigskip
\bigrm
 
{\bigsl APL} has been around for ages and back when the input method was a
modified IBM Selectric typewriter with a custom type ball, it was cool.  But I
always found it to be a Royal Pain$^{{\rm TM}}$ remembering the
finger-tangling combination of ctrl-, shift-, and alt- keys needed on modern
keyboards to get dominoes, quotequads, and whatnot.

J, at least, avoids all that, but has its own weirdnesses

\input general-ops-table
\input comparison-ops-table
\input trigonometric-ops-table
\input matrix-vector-ops-table





\bye
